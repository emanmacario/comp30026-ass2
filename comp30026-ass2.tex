\documentclass[12pt]{article}
\usepackage{graphicx}
\usepackage{xspace}
\usepackage{color, colortbl}
\usepackage{amssymb}
\usepackage{amsmath}
\usepackage{mathtools}
\pagestyle{empty}
\definecolor{Gray}{gray}{0.9}
\textwidth      165mm
\textheight     240mm
\topmargin      -18mm
\oddsidemargin  -2mm
\evensidemargin 2mm
\newcommand{\impl}{\mathbin{\Rightarrow}}
\newcommand{\biim}{\mathbin{\Leftrightarrow}}
\newcommand{\id}[1]{\mbox{\textit{#1}}}
\newcommand{\ma}{\mathsf{a}}
\newcommand{\mb}{\mathsf{b}}
\newcommand{\mc}{\mathsf{c}}
\newcommand{\deriv}{\ensuremath{d_1}}
\renewcommand{\theenumi}{\alph{enumi}}

\DeclarePairedDelimiter\Floor\lfloor\rfloor
\DeclarePairedDelimiter\Ceil\lceil\rceil

\author{Emmanuel Macario - 831659}
\title{COMP30026 Models of Computation Assignment 2}
\date{October, 2018}

\begin{document}

\maketitle

% CHALLENGE 4
\subsection*{Challenge 4}

\begin{enumerate}
\item
Let $D$ be a DFA $(Q,\Sigma,\delta,q_0,F)$ that recognises a regular language $R$. We can transform 
$D$ into an NFA $N$ that recognises $\id{drop}(\ma,R)$ by replacing every transition involving the symbol
$\ma$ with an epsilon transition.

\bigskip
\noindent
More formally, we can define $N$ to be the five-tuple $N=(Q,\Sigma_\epsilon \setminus \{\ma\},\delta',q_0,F)$
with the transition function,

\[
  \delta'(q, x) =
  \begin{cases}
  	\{\delta (q, x)\}     & \text{if $q \in Q$ and $x \in \Sigma \setminus \{\ma\}$} \\
      \{\delta (q, \ma)\} & \text{if $q \in Q$ and $x = \epsilon$} \\
  \end{cases}
\]

\bigskip
\noindent
Hence, since $\id{drop}(\ma,R)$ can be recognised by NFA $N$, then $\id{drop}(\ma,R)$ is also a regular language.

\item $L=\{a^m b^n c^n \mid m, n \geq 0\}$

\end{enumerate}
% CHALLENGE 5
\subsection*{Challenge 5}

% CHALLENGE 6
\subsection*{Challenge 6}






\end{document}